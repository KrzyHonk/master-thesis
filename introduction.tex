\chapter{Introduction}
\label{cha:introduction}
\section{Motivation}
Business process management plays an important part in modern corporation and enterprise management. Business process models can be used as a documentation for work-flow implementation, partial automation or optimization of process.  In order to analyse and optimize the process (for example, eliminate bottlenecks or analyse time consumption), a structured, formal model is required. There are several tools which provides formalized, standardized business models. One of the most popular standards is BPMN (\emph{Business Process Model and Notation})~\cite{BPMN20}, a graphical representation standard maintained by OMG (\emph{Object Management Group}). BPMN tools (such as Signavio Process Editor\footnote{\url{www.signavio.com}, last access: \onlineAccess}) can support process simulation and makes analysis of business process easier.\\
Usually, some knowledge about the processes and work-flows exists in form of human knowledge and textual documentation. Manual extraction of a process model from technical documentation (textual description) is a time-consuming process. Since every business enterprise must constantly improve its services, their process models must be frequently updated. Moreover, manually designed models can be different, depending on creator (usually a business analyst) experience and knowledge.\\
In an attempt to resolve this problem, several approaches of automatic business process models generation have been proposed in recent years. An effective way of machine-aided transformation from a semi-formal document into a process model can provide a significant savings in time, additionally making maintenance of formal process models and documentation easier. Furthermore, an automatic tool for model extraction can be very useful for people, who do not have the sufficient knowledge and expertise in process modelling field.

\section{Thesis goal}
The aim of this thesis is to present the current state-of-the-art in the field of process model extraction from textual description and to provide a prototype of simple tool for generating a BPMN diagram from textual description.\\
The method for extracting business process from natural language text, implemented in prototype, is based on syntactic analysis of given natural text and extracting Subject-Verb-Object construct (SVO), which can be later transformed into process activities. This method is enhanced with semantic analysis of the text, which allows to filter out unnecessary SVO constructs and transform them into valid activities names. The implemented method was tested on a set of natural language business process descriptions, gathered from different BPMN tutorials and academic sources (textbooks, courses).

\section{Structure of the thesis}
Section~\ref{cha:background} focuses on providing basic information about Business Process Management, BPMN and NLP (\emph{Natural Language Processing}), introducing basic concepts and ideas. NLP part will also describe the tools used in this approach, basically the SpaCy parser and the underlying technology of this tool. Section~\ref{cha:implementation} shortly describes the functional requirements of project and presents the business process extraction and BPMN diagram generation method details, describing step-by-step the algorithms used in prototype. Section~\ref{cha:validation} shows the experimental results of applying the implemented prototype for the obtained test data. Section~\ref{cha:conclusion} recapitulates the whole thesis and highlights the future possible improvements.