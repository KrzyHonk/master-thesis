\chapter{Conclusion}
\label{cha:conclusion}
\section{Summary}
In this thesis, a solution for the problem of automatic generation of business process models from natural language description was presented. An effective, machine-aided transformation from a semi-formal or informal document into a process model might be more time-efficient in comparison to the manual generation of process model diagram. In addition, maintenance of formal process models and documentation might become easier, especially for people, who do not have the sufficient knowledge and expertise in process modelling field.\\
The proposed solution is based on syntactic analysis of business process description and extracting Subject-Verb-Object constructs (SVO), which can be later transformed into process elements. The presented method is enhanced with semantic analysis of the text, which allows to filter out unnecessary SVO constructs. The methodology presented in this thesis is divided the following steps:
\begin{enumerate}
	\item Participants extraction -- a sentence from a given description is analysed, and the information about possible participants in process are extracted,
	\item Subject-verb-object constructs extraction -- a sentence from given description is analysed in search of basic SVO constructs,
	\item Gateway keywords search -- a process description is analysed in search of the keywords that signalizes the presence of logical gateways,
	\item Intermediate process model generation -- an intermediate process model is created from the acquired data,
	\item BPMN diagram generation -- a BPMN diagram is generated from the intermediate process model.
\end{enumerate}
A prototype of the proposed method was implemented using Python programming language and SpaCy library, which provides many useful Natural Language Processing tools (syntax parser, WordNet lexical database API). This prototype was tested against a test set of natural language business process descriptions, gathered from a few academic sources. The validation of test results shows that proposed method is able to create a complete process model, but a few limitations of presented approach were highlighted (only task and gateway elements used in generated models, limited effectiveness of gateway extraction, problems with extracting invalid activities from description, lack of anaphora resolution).

\section{Conclusion. Further works}
The proposed method of generating process model from natural language description provides some basic information about the described process in the form of BPMN diagram. It is not able to extract more complex constructs and is only able to handle basic elements of BPMN standard. Dealing with the method limitations listed in the previous section provides a direction for further work on improving the proposed solution. Enhancing the process models, generated using proposed method, with additional BPMN elements (such as intermediate events, pools and lanes); extractions more information about conditional flows and gateways; adding anaphora resolution to identify real actors in process -- these improvements are possible to achieve with further work on proposed method.\\
On the other hand, extracting business process information from natural language description proved to be a difficult task. The problem lies in the nature of natural language -- the natural language processing methods have a limited knowledge of semantics and context of words. It is almost impossible to discard these parts of description, which do not provide any information about process execution. Thus, it might be a better idea to implement a semi-automatic solution, which requires some amount of human aid or to abandon natural language description for some sort of standardized format or structured natural language description. For example, there are a few publications that focus on the subject of translating the SBVR notation into process models~\cite{sbvr-automat}. By using the SBVR as a process description format it is possible to define basic syntax and structure for description, which can be easier to process.\\
Nevertheless, generating a process model from natural language description is a challenging task. The proposed method provides limited results, but with further development, it might be possible to provide overcome the highlighted problems and improve the results significantly.